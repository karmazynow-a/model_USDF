
%----------------------------------------------------------------------------------------
%	PREAMBUŁA
%----------------------------------------------------------------------------------------

\documentclass[12pt]{article}
\usepackage[polish]{babel}
\usepackage{polski}
\usepackage[utf8]{inputenc}
%\usepackage[T1]{fontenc}
\usepackage{amsmath}
\usepackage{graphicx}
\usepackage{fancyhdr}
\usepackage{float}
\usepackage{graphicx}
\usepackage{hyperref}
\usepackage{verbatim}

\usepackage{subfig}

\usepackage{color} %red, green, blue, yellow, cyan, magenta, black, white
\definecolor{mygreen}{RGB}{28,172,0} % color values Red, Green, Blue
\definecolor{mylilas}{RGB}{170,55,241}


\title{Sprawozdanie}
\author{Aleksandra Poręba}

\graphicspath{{static/}} 

\makeatletter
\let\thetitle\@title
\let\theauthor\@author
\let\thedate\@date
\makeatother


%----------------------------------------------------------------------------------------
%	STRONA TYTUŁOWA
%----------------------------------------------------------------------------------------
\begin{document}
\begin{center}
\textsc{\normalsize Wydział Fizyki i Informatyki Stosowanej}\\[2.0cm] 
\includegraphics[scale = 1]{logo.png}\\[1cm] 
%\textsc{\Large Modelowanie Procesów Fizycznych}\\[0.4cm] 


\textsc{\Large Sprawozdanie}\\[0.4cm]
{ \huge \bfseries \LARGE{Projekt 2: Sznajd 2D} }\\[1cm] 

\flushright \Large Aleksandra Poręba \\ nr. indeksu 290514

\vfill 

\center{\today}


\pagebreak 

\end{center}

%----------------------------------------------------------------------------------------
%	SPIS TREŚCI
%----------------------------------------------------------------------------------------
%\tableofcontents
%\pagebreak

%----------------------------------------------------------------------------------------
%	ZAWARTOŚĆ
%----------------------------------------------------------------------------------------

\pagestyle{fancy}
\fancyhf{}

\rhead{\theauthor}
\lhead{\thetitle}
\cfoot{\thepage}

\section{Opis projektu}

\section{Ciekawe reguły}

\section{Używanie aplikacji}
% o toggle
Możliwa jest również modyfikacja początkowej gęstości siatki oraz jej rozmiaru, za pomocą z menu ''Parametry wejściowe''. Poniżej menu znajduje się wykres prezentujący gęstość życia w symulacji dla kolejnych iteracji. Po prawej stronie aplikacji znajduje się siatka, na której prezentowane jest działanie automatu.

Symulacją można sterować za pomocą przycisków ''START'', ''RESET'', ''STOP'' oraz ''STEP''. Gdy zostaną zmienione parametry, siatka jest automatycznie resetowana - siatka która zostanie zresetowana zmienia kolory na poszarzałe.

Do stworzenia strony zostały użyte technologie HTML + JS + CSS oraz ZingChart do rysowania wykresu.

\end{document}
